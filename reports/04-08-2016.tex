%% LyX 2.1.4 created this file.  For more info, see http://www.lyx.org/.
%% Do not edit unless you really know what you are doing.
\documentclass[english]{article}
\usepackage[T1]{fontenc}
\usepackage[latin9]{inputenc}

\makeatletter
%%%%%%%%%%%%%%%%%%%%%%%%%%%%%% User specified LaTeX commands.
\usepackage[style=authoryear,backend=biber]{biblatex}

\makeatother

\usepackage{babel}
\begin{document}

\title{MEMO-H-504\\
Report of the Meeting of 04/08/16}


\author{Andrea Morciano}

\maketitle
\noindent The people present at the meeting were me, my advisor prof.
Stijn Vansummeren and my supervisor Martin Ugarte.\medskip{}


\noindent I did a presentation divided into three parts:
\begin{itemize}
\item \noindent my work so far;
\item \noindent the current challenges in my work;
\item \noindent Q\&A.
\end{itemize}

\section{My Work}

I wrote two introductory chapters of my thesis. The first one is a
generic overview of Information Extraction and its techniques. The
second is a description of SystemT and its query language AQL. For
the most part of this chapter I describe the formal model for AQL's
core, based on document spanners. My paper is written in Lyx, that
is a document editor based on Latex. It can output PDF documents and
Latex files as well. Martin installed this software on his machine,
in order to be able to directly modify my document.\medskip{}
For what concerns the practical part, I presented the approach that
I chose to evaluate NFAs: a virtual machine that advances all the
branches of execution in lockstep, and that easily supports submatch
extraction. I provided an example, on which we discussed on how this
runtime could realize the execution of a vset-automaton: it should
output all the matches it finds, not only the longest one, which is
determined according to the greedy leftmost policy. I also showed
how I transform a vset-automaton into a vset path union.


\section{Current Challenges}

We further discussed how to execute a vset automaton. There are two
main possibilities: 
\begin{enumerate}
\item using a vset path union (possibly optimized in some way) which has
the property to guarantee that, in each path, each span variable is
opened and closed exactly once;
\item executing the vset-automaton directly. This requires a modification
of the runtime, because each thread needs to keep track of the available
variables, in order to die if there is an atttempt to open a variable
more than once or if upon match some available variables remain (preventing
the match).
\end{enumerate}
We agreed that exploring the convenience of the two approaches could
be a contribution to the research on the topic. Also trying different
runtimes (e.g. OBDDs) could be interesting.\medskip{}
We also spoke about how to perform the intersection of two vset automata.
This operation is needed to implement the $\bowtie$ operator. I presented
some software libraries that can interesect ordinary FSA, but we agreed
that they don't understand vset-automata, that are equipped which
special $\varepsilon$-transitions, i.e. variable operations. Thus,
implementing FSA intersection from scratch would be better. Finally,
we briefly discussed on how to translate a vset-automaton in the format
required by the runtime.


\section{Q\&A}

I asked an opinion on the priority of the tasks I have to perform
and we agreed to concentrate on the ways of executing a vset-automaton
first. We then discussed how to obtain production code using LMS and
LLVM. We also talked about the deployment of SystemT on the machine
of the department, although prof. Vansummeren noticed that a performance
comparison between my system and BigInsight (the platform containing
SystemT) would be of little use, as BigInsights is a much more sophsticated
system. Anyway we agreed to put me in contact with the administrator
of the machine.\medskip{}
I stated that I want to present my thesis at the second session. This
implies using LMS in my work, and having performed experiments with
the system by the beginning of July.
\end{document}
