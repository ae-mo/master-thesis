%% LyX 2.1.4 created this file.  For more info, see http://www.lyx.org/.
%% Do not edit unless you really know what you are doing.
\documentclass[11pt,american,draftfoot]{ulb_thesis}
\usepackage[T1]{fontenc}
\usepackage[latin9]{inputenc}
\usepackage{amsmath}
\usepackage{amssymb}
\usepackage{stmaryrd}

\makeatletter
%%%%%%%%%%%%%%%%%%%%%%%%%%%%%% Textclass specific LaTeX commands.
 \input{ulb_ntheorem.std}

\makeatother

\usepackage{babel}
\begin{document}
$\textrm{SVOps}^{\vdash}=\left\{ x\vdash\mid x\in\textrm{SVars}\right\} $

$\textrm{SVOps}^{\dashv}=\left\{ x\dashv\mid x\in\textrm{SVars}\right\} $

SVOps$=\textrm{SVOps}^{\vdash}\cup\textrm{SVOps}^{\dashv}$
\begin{definition}
An extended \emph{variable-set automaton} (or \emph{eVset-automaton})
is a tuple $\left(Q,q_{0},q_{f},\delta\right)$, where:
\begin{itemize}
\item $Q$ is a finite set of states;
\item $q_{0}\in Q$ is the initial state;
\item $q_{f}\in Q$ is an accepting state;
\item $\delta=\delta^{char}\cup\delta^{op}$ is a finite transition relation
consisting of triples, where:

\begin{itemize}
\item $\delta^{char}=\left\{ \left(q,\sigma,q'\right)\mid q,q'\in Q,\sigma\in\varSigma\right\} $,
whose elments are called character transitions;
\item $\delta^{op}=\left\{ \left(q,S,q'\right)\mid q,q'\in Q,S\subseteq\left(\textrm{SVOps}\right)\right\} $,
whose elements are called operation transitions .
\end{itemize}
\end{itemize}
\end{definition}


\begin{definition}
Given a transition $t=\left(q,S,q'\right)\in\delta^{op}$ in an eVset-automaton
$A$, and an ordering $\varphi$ on the elements of $\textrm{SVOps}\left(A\right)$,
we define:
\begin{itemize}
\item $\textrm{LOps}_{\varphi}\left(t\right)$ as the list $o_{1},\ldots,o_{\left|S\right|}$
of operations belonging to $S$, ordered according to $\varphi$;
\item $\textrm{Ops}\left(t\right)=S$.
\end{itemize}
\end{definition}


\begin{definition}
\noindent Given an extended vset-automaton $A=\left(Q,q_{0},q_{f},\delta\right)$
and a pair of states $q,q'\in Q$, a path $p$ between $q$ and $q'$
in $A$ is a sequence of transitions $t_{1},$$\ldots,t_{n}\in\delta$,
such that:
\begin{itemize}
\item the source state of $t_{1}$ is $q$;
\item the destination state of $t_{n}$ is $q'$;
\item \noindent for every pair $t_{i},t_{i+1}$, the destination state of
$t_{i}$ equals the source state of $t_{i+1}$.
\end{itemize}
We refer to the set of paths in $A$ as $\textrm{Paths}\left(A\right)$.
\end{definition}


\begin{definition}
\noindent Given an eVset-automaton $A=\left(Q,q_{0},q_{f},\delta\right)$
a path $p=t_{1},\ldots,t_{n}$ in $A$ is complete if it is between
$q_{0}$ and $q_{f}$.
\end{definition}


\begin{definition}
Given a path $p=t_{1},\ldots,t_{n}$ in an eVset-automaton $A$, and
considering the sublist $p^{op}=t_{j},\ldots t_{k}$, where $\forall t_{j}$,
$t_{j}\in\delta^{op}$ and an ordering $\varphi$ on the elements
of $\textrm{SVOps}\left(A\right)$, we define
\begin{itemize}
\item the list $\textrm{LOps}_{\varphi}\left(p\right)$ as:
\[
o_{j,1},\ldots,o_{j,l_{j}},\ldots,o_{k,1},\ldots,o_{k,l_{k}}
\]
where $\forall j,m$, $o_{j,m}$$\in S_{j}$, and the order of the
operations belonging to the same $S_{j}$ is fixed according to $\varphi$;
\item the set $\textrm{Ops}\left(p\right)$ as the set of operations appearing
in $\textrm{LOps}_{\varphi}\left(p\right)$.
\end{itemize}
If $p^{op}=p$, we say that $p$ is an operation-only path.
\end{definition}


\begin{definition}
\noindent \label{well-behaved}A (extended) variable-set automaton
$A$ is well-behaved if, for every complete path $p$ in $A$ we have:
\begin{enumerate}
\item for every $o\in\textrm{SVOps}\left(A\right)$, $o$ appears exactly
once in $\textrm{LOps}\left(p\right)$;\label{enu:well-behaved-1,}
\item for every pair of operations $x\vdash$,$\dashv x$$\in\textrm{SVOps}\left(A\right)$,
$x\vdash$ appears before $\dashv x$ in $\textrm{LOps}\left(p\right)$.
\end{enumerate}
\end{definition}

\noindent 
\begin{definition}
Given an eVset-automaton $A$, $A$ is pruned if for every state $q$
in $A$, there is a path from $q_{0}$ to $q$, and a path from $q$
to $q_{f}$ in $A$.
\end{definition}

\noindent 
\begin{proposition}
Given a well-behaved eVset-automaton $A$, there exists a pruned well-behaved
eVset automaton $A'$ which is equivalent to $A$.\end{proposition}

\begin{proof}
To obtain $A'$, it is sufficient to remove from $A$ those states
that don't participate in any complete path in $A$, and all the transitions
involving them. Because of our construction, $A'$ is well-behaved,
as it now contains only the complete paths in $A$. To see that $A'$
is equivalent to $A$, consider a state $q$ of $A$ that is not present
in $A'$. $q$ could never occur in a configuration of an accepting
run of $A$, because either it is unreachable from the initial state
of $A$, either it couldn't reach the final state of $A$. Thus, given
an accepting run of $A$, there will be a corresponding accepting
run of $A'$. $A'$ cannot have an accepting run that can't be put
in correspondence with an accepting run of $A$ because we do not
add any transitions to $A'$ that aren't present in $A$.
\end{proof}

\noindent 
\begin{corollary}
\label{same-ops}Given a pruned well-behaved eVset-automaton $A$,
and two states $q,q'$ in $A$, for every pair of paths $p,p'$ between
$q$ and $q'$ in $A$, $\textrm{Ops}\left(p\right)=\textrm{Ops}\left(p'\right)$.\end{corollary}

\begin{proof}
Since $A$ is pruned, there surely exists a path $p^{0}$ between
$q_{0}$ and $q$. For the same reason, there exists a path $p^{f}$
between $q'$ and $q_{f}$. So $q$ and $q'$ appear in two complete
paths, which we may call $p^{0}pp^{f}$ and $p^{0}p'p^{f}$, that
differ only in the subpaths between $q$ and $q'$. If the statement
was false, then one of $p^{0}pp^{f}$ and $p^{0}p'p^{f}$ wouldn't
satisfy the requirement \ref{enu:well-behaved-1,} of \ref{well-behaved},
making $A$ not well-behaved, a contraddiction.
\end{proof}

\noindent 
\begin{definition}
Given a eVset-automaton $A$, $A$ is operation-closed if, for every
pair of states $q,q'$ in $A$, whenever there exists an operation-only
path $p$=$t_{1},\ldots,t_{n}$ between $q$ and $q'$, then there
exists a transition in $A$ of the form $\left(q,\bigcup_{1\leq i\leq n}\textrm{Ops}\left(t_{i}\right),q'\right)$.
\end{definition}

\noindent 
\begin{proposition}
Given a well-behaved eVset-automaton $A$, there exists an operation-closed
well-behaved eVset-automaton $A'$ that is equivalent to $A$.\end{proposition}

\begin{proof}
Let us consider $A$ to be of the form $\left(Q,q_{0},q_{f},\delta\right)$.
Without loss of generality, we can assume that $A$ is pruned. Then
we can take $A'=\left(Q',q_{0}',q_{f}',\delta'\right)$, where:
\begin{itemize}
\item $Q'=Q$;
\item $q_{0}'=q_{0}$;
\item $q_{f}'=q_{f}$;
\item $\delta'=\delta\cup\left\{ \left(q,\bigcup_{1\leq i\leq n}\textrm{Ops}\left(t_{i}\right),q'\right)\mid\exists p=t_{1},\ldots,t_{n}\in\textrm{Paths}\left(A\right):p\textrm{ is between \ensuremath{q} and \ensuremath{q'}, \ensuremath{p} is operation-only}\right\} $.
\end{itemize}
This construction does nothing but including the transitions that
are missing from $A$ to be operation-closed to $A'$. We now show
that $A'$ is equivalent to $A$. It is easy to see that, for every
string $\mathbf{s}\in\varSigma^{\ast}$, $A\left(\mathbf{s}\right)\subseteq A'\left(\mathbf{s}\right)$,
as we include all transitions of $A$ in $A'$. To see that $A'\left(\mathbf{s}\right)\subseteq A\left(\mathbf{s}\right)$,
consider a pair of states $q,q'\in Q$ and let us assume there exists
an operation-only path $p'=t$ between $q$ and $q'$ in $A$', where
$t$ is a transition that is not in $A$. By construction of $A'$,
there exists in $A$ an operation-only path $p$ between $q$ and
$q'$, and$\textrm{LOps}\left(p'\right)=\textrm{LOps}\left(p\right)$.
Thus, $p'$ is equivalent to $p$. This also shows that $A'$ is well-behaved.
\end{proof}

\noindent 
\begin{definition}
Given two well-behaved operation-closed eVset-automata $A=\left(Q_{A},q_{A}^{0},q_{A}^{f},\delta_{A}\right)$
and $B=\left(Q_{B},q_{B}^{0},q_{B}^{f},\delta_{B}\right)$, their
product is a eVset-automaton $C=\left(Q,q^{0},q^{f},\delta\right)$,
where:
\begin{itemize}
\item $Q=Q_{A}\times Q_{B}$;
\item $q^{0}=\left\langle q_{A}^{0},q_{B}^{0}\right\rangle $;
\item $q^{f}=\left\langle q_{A}^{f},q_{B}^{f}\right\rangle $;
\item $\delta$ has the following transitions:

\begin{itemize}
\item $\left(\left\langle q_{A},q_{B}\right\rangle ,\sigma,\left\langle q'_{A},q'_{B}\right\rangle \right)$
whenever $\sigma\in\varSigma$, $\left(q_{A},\sigma,q'_{A}\right)\in\delta_{A}$
and $\left(q_{B},\sigma,q'_{B}\right)\in\delta_{B}$;
\item $\left(\left\langle q_{A},q_{B}\right\rangle ,S_{A}\cup S_{B},\left\langle q'_{A},q'_{B}\right\rangle \right)$
whenever $\left(q_{A},S_{A},q'_{A}\right)\in\delta_{A}$, $\left(q_{B},S_{B},q'_{B}\right)\in\delta_{B}$
and $\left(\textrm{SVOps}\left(A\right)\cap\textrm{SVOps}\left(B\right)\right)\cap S_{A}=\left(\textrm{SVOps}\left(A\right)\cap\textrm{SVOps}\left(B\right)\right)\cap S_{B}$;
\item $\left(\left\langle q_{A},q_{B}\right\rangle ,S_{A},\left\langle q'_{A},q{}_{B}\right\rangle \right)$
whenever $\left(q_{A},S_{A},q'_{A}\right)\in\delta_{A}$ and $\left(\textrm{SVOps}\left(A\right)\cap\textrm{SVOps}\left(B\right)\right)\cap S_{A}=\textrm{�}$;
\item $\left(\left\langle q_{A},q_{B}\right\rangle ,S_{B},\left\langle q{}_{A},q'_{B}\right\rangle \right)$
whenever $\left(q_{B},S_{B},q'_{B}\right)\in\delta_{B}$ and $\left(\textrm{SVOps}\left(A\right)\cap\textrm{SVOps}\left(B\right)\right)\cap S_{B}=\textrm{�}$.
\end{itemize}
\end{itemize}
We may write $C=A\times B$.
\end{definition}

\noindent 
\begin{definition}
Given two well-behaved operation-closed eVset-automata $A$ and $B$,
and given an eVset-automaton $C$ such that $C=A\times B$, the projection
of a transition $t$ in $C$ over $A$ is a transition $t'$ in $A$
defined as:
\begin{itemize}
\item $\left(q_{A},\sigma,q'_{A}\right)$ if $t=\left(\left\langle q_{A},q_{B}\right\rangle ,\sigma,\left\langle q'_{A},q'_{B}\right\rangle \right)$;
\item $\left(q_{A},S_{A},q'_{A}\right)$ if $t=\left(\left\langle q_{A},q_{B}\right\rangle ,S_{A}\cup S_{B},\left\langle q'_{A},q'_{B}\right\rangle \right)$
or $t=\left(\left\langle q_{A},q_{B}\right\rangle ,S_{A},\left\langle q'_{A},q{}_{B}\right\rangle \right)$.
\end{itemize}
We may write $t'=\pi_{A}\left(t\right)$. The definition is symmetric
for the projection over $B$.
\end{definition}

\noindent 
\begin{definition}
Given two well-behaved operation-closed eVset-automata $A$ and $B$,
and given an eVset-automaton $C$ such that $C=A\times B$, the projection
of a path $p=t_{1},\ldots,t_{n}$ in $C$ over $A$ is a path $p'$
in $A$ defined as:
\[
t'_{1},\ldots t'
\]

\end{definition}

\noindent 
\begin{theorem}
Given two well-behaved operation-closed eVset-automata $A$ and $B$,
and given an eVset-automaton $C$ such that $C=A\times B$, then $\left\llbracket C\right\rrbracket =\left\llbracket A\right\rrbracket \bowtie\left\llbracket B\right\rrbracket $.\end{theorem}

\begin{proof}
To show that $\left\llbracket C\right\rrbracket \subseteq\left\llbracket A\right\rrbracket \bowtie\left\llbracket B\right\rrbracket $,
we can easily decompose a run of $C$ on a string $\mathbf{s}$ into
two consistent runs of $A$ and $B$. To show that $\left\llbracket A\right\rrbracket \bowtie\left\llbracket B\right\rrbracket \subseteq\left\llbracket C\right\rrbracket $,
consider a string $\mathbf{s}$, a $\mathbf{s}$-tuple $\mu_{A}\in A\left(\mathbf{s}\right)$
and a $\mathbf{s}$-tuple $\mu_{B}\in B\left(\mathbf{s}\right)$ that
agrees with $\mu_{A}$ on common variables. Given the $\mathbf{s}$-tuple
$\mu$ that yelds all the variable assignments of $\mu_{A}$ and $\mu_{B}$.
We need to find a run $\rho$ of $C$ on $\mathbf{s}$ that returns
$\mu$. Let us call $\rho_{A}\in\textrm{ARuns}\left(A,\mathbf{s}\right)$
and $\rho_{B}\in\textrm{ARuns}\left(B,\mathbf{s}\right)$ the runs
that return $\mu_{A}$ and $\mu_{B}$, respectively. We can obtain
$\rho$ by combining $\rho_{A}$ and $\rho_{B}$. To do that, we first
need to ensure that $\rho_{A}$ and $\rho_{B}$ are consistent on
the order of the variable operations they perform. Since $A$ and
$B$ are well-behaved and operation-closed, $\rho_{A}$ and $\rho_{B}$
can always be selected so that they are consistent.
\end{proof}

\noindent 
\begin{lemma}
\noindent Given an extended varable-set automaton $A=\left(Q,q_{0},q_{f},\delta\right)$,
$A$ can be converted into an equivalent variable-set automaton $A'$
in polynomial time in a well-behavedness preserving manner.\end{lemma}

\begin{proof}
Without loss of generality, we consider an ordering of the symbols
in $\textrm{SVOps}\left(A\right)$ of the following form:
\[
x\vdash,\ldots,y\vdash,x\dashv,\ldots,y\dashv
\]
 In this ordering, all insertion operations come before the deletion
operations. We define $o\prec o'$, $o,o'\in\textrm{SVOps}\left(A\right)$,
if $o$ comes before $o'$ (\emph{not} if they are equal) in the chosen
ordering. Consider $A'=$$\left(Q',q_{0}',q_{f}',\delta'\right)$,
with $\textrm{SVars}\left(A'\right)=\textrm{SVars}\left(A\right)$,
whose components are defined as follows:
\begin{itemize}
\item $Q'=Q\cup Q^{\textrm{ops}}\cup Q^{\textrm{�}}$, where:

\begin{itemize}
\item $Q^{\textrm{ops}}=\left\{ q_{q',o,q''}\mid\exists\left(q',S,q''\right)\in\delta:o\in S\right\} $;
\item $Q^{\textrm{�}}=\left\{ q_{\textrm{q',�,q''}}\mid\exists\left(q',\textrm{�},q''\right)\in\delta\right\} $;
\end{itemize}
\item $q_{0}'=q_{0};$
\item $q_{f}'=q_{f}$;
\item $\delta'=\left(\delta\setminus\delta^{S}\right)\cup\delta^{\textrm{ops}}\cup\delta^{\textrm{�}}\cup\delta^{\varepsilon}$,
where:

\begin{itemize}
\item $\delta^{S}=\left\{ \left(q,S,q'\right)\in\delta\right\} ;$
\item $\delta^{\textrm{ops}}=\left\{ \left(q_{q',o,q''},o,q_{q',o',q''}\right)\mid\exists\left(q',S,q''\right)\in\delta:\left(o,o'\epsilon S\land o\prec o'\land\forall o''\in S:o\nprec o''\nprec o'\right)\right\} \cup\left\{ \left(q_{q',o,q''},o,q''\right)\mid\exists\left(q',S,q''\right)\in\delta:\left(o\epsilon S\land\forall o'\in S:o\nprec o'\right)\right\} $;
\item $\delta^{\textrm{�}}=\left\{ \left(q_{\textrm{q',�,q''}},\varepsilon,q''\right)\mid\exists\left(q',\textrm{�},q''\right)\in\delta\right\} ;$
\item $\delta^{\varepsilon}=\left\{ \left(q,\varepsilon,q'\right)\mid\left(\exists\left(q',o,q''\right)\in\delta^{\textrm{ops}}:\forall\left(q',o',q''\right)\in\delta^{\textrm{ops}}:o'\nprec o\right)\lor\left(\exists\left(q',\varepsilon,q''\right)\in\delta^{\textrm{�}}\right)\right\} $.
\end{itemize}
\end{itemize}
This construction expands the transitions of $A$ that are labeled
with a set of variable operations into a sequence of transitions performing
one operation at a time, taking care of putting the insertion operations
before the deletion ones, thus preserving well-behavedness. It also
substitutes transitions labeled with the empty set with ordinary $\varepsilon-$transitions.
To prove equivalence between $A$ and $A'$ it is sufficient to notice
that for every string $\mathbf{s}$, every run $\rho=c_{0},\ldots,c_{m}$$\in\textrm{ARuns}\left(A,\mathbf{s}\right)$
can be put in correspondence with a run $\rho'\in\textrm{ARuns}\left(A',\mathbf{s}\right)$.
Indeed, for every pair of consecutive configurations $c_{j}=\left(q_{j},V_{j},Y_{j},i_{j}\right)$,
$c_{j+1}=\left(q_{j+1},V_{j+1},Y_{j+1},i_{j+1}\right)$:
\begin{itemize}
\item if $i_{j}=i_{j+1}$ and $\left(q,S,q'\right)$$\in\delta$:

\begin{itemize}
\item if $V_{j}\neq V_{j+1}$, replace $c_{j}$, $c_{j+1}$ with a sequence
$c_{j,1},\ldots,c_{j,k}$ of configurations, where:

\begin{itemize}
\item $c_{j,1}=c_{j}$;
\item $c_{j,k}=c_{j+1}$; 
\item $c_{j,2}=\left(q_{j,2},V_{j,1},Y_{j,1},i_{j}\right)$ ($\left(q_{j,1},\varepsilon,q_{j,2}\right)\in\delta^{\varepsilon})$;
\item for every pair $c_{j,l}=\left(q_{j,l},V_{j,l},Y_{j,l},i_{j}\right),c_{j,l+1}=\left(q_{j,l+1},V_{j,l+1},Y_{j,l+1},i_{j}\right)$,
different from $c_{j,1},c_{j,2}$, there exists a transition $\left(q_{j,l},o,q_{j,l+1}\right)\in\delta^{\textrm{ops}}$
such that $o\in S$ and $V_{j,l+1},Y_{j,l+1}$ are obtained from $V_{j,l},Y_{j,l}$
according to $o$;
\end{itemize}
\item if $V_{j}=V_{j+1}$, replace $c_{j}$, $c_{j+1}$with $c_{j,1},c_{j,2},c_{j,3}$,
where:

\begin{itemize}
\item $c_{j,1}$ and $c_{j,2}$ are as in the previous case;
\item $c_{j,3}=c_{j+1}$;
\item there exists a transition $\left(q_{j,2},\textrm{�},q_{j,3}\right)\in\delta^{\textrm{�}}$
and $V_{j,3}=V_{j,l}$, $Y_{j,3}=Y_{j,2}$;
\end{itemize}
\end{itemize}
\item if $i_{j+1}=i_{j}+1$, do nothing.
\end{itemize}
After having applied this modifications to $\rho$, the obtained run
$\rho'$ will be a valid run of $A'$ on $\mathbf{s}$ and it will
be accepting if and only if $\rho$ was accepting, in which case it
will return the same $\mathbf{s}-$tuple returned by $\rho$. This
is because the only occasions where we actually modify $\rho$ is
when we are performing variable operations between a configuration
and the next one. In this case, we are not advancing the string pointer.
The new series of configurations we put in place of the original pair
doesn't advance the pointer either, and it performs the exact variable
operations that were performed before the substitution, insertions
first, thanks to the way we constructed $A'$. Thus, for every pair
of configurations $c_{j},c_{j+1}\in\rho$ that we replace, $c_{j+2}$
won't see any difference. Also notice how the replaced sequence $c_{j,1},\ldots,c_{j,k}$
is unique, and the sequence of operations from $c_{j,1}$ to $c_{j,k}$
respect the operation ordering that we chose. It can also easily be
verified that well-behavedness is preserved.

\noindent It remains to show that the construction of $A'$ can be
carried in polynomial time. Let us refer to the size of $A$ as $n$.
We consider $n=\left|Q\right|+\left|\delta\right|+\left|\textrm{SVars}\left(A\right)\right|$.
We can reasonably assume that the operations of adding a new state,
adding a new transition, removing a state and removing a transition
take constant time. Thus, expanding a single transition of $A$ takes
$O\left(\left|\textrm{SVars}\left(A\right)\right|\right)=O\left(n\right)$
time. Expanding every transition that is needed will then take $O\left(\left|\textrm{SVars}\left(A\right)\right|\cdot\left|\delta\right|\right)=O\left(n^{2}\right)$
time.

\medskip{}
\end{proof}

\begin{definition}
\noindent Given an extended vset-automaton $A$, the epsilon-closure
of $A$ is an extended vset-automaton $A'=\left(Q',q_{0}',q_{f}',\delta'\right)$
, where:
\begin{itemize}
\item $Q'=Q$;
\item $q_{0}'=q_{0}$;
\item $q_{f}'=q_{f}$;
\item $\delta\subseteq\delta'$;
\item for every sequence of triples $\left(q_{1},S_{1},q'_{1}\right),\ldots,\left(q_{n},S_{n},q'_{n}\right)\in\delta$
such that $\forall i$,$1\leq i<n$ we have $q'_{i}=q_{i+1}$, $\delta'$
contains a triple $\left(q_{1},\bigcup_{i}S_{i},q_{n}\right)$ and
no additional triples that don't belong to $\delta$.
\end{itemize}
\end{definition}


\begin{theorem}
\noindent Given two extended vset-automata $A$ and $B$ that are
union-compatible, a third vset-automaton $C$ can be produced in polynomial
time such that $\left\llbracket C\right\rrbracket =\left\llbracket A\right\rrbracket \cup\left\llbracket B\right\rrbracket $,
in a well-behavedness preserving manner.\end{theorem}

\begin{proof}
Let us consider $A$ to be of the form $\left(Q^{A},q_{0}^{A},q_{f}^{A},\delta^{A}\right)$
and $B$ of the form $\left(Q^{B},q_{0}^{B},q_{f}^{B},\delta^{B}\right)$.
We can take $C=\left(Q^{C},q_{0}^{C},q_{f}^{C},\delta^{C}\right)$,
where:
\begin{itemize}
\item $Q^{C}=Q^{A}\cup Q^{B}$;
\item $q_{0}^{C}=q_{0}^{A}$;
\item $q_{f}^{C}=q_{f}^{A}$;
\item $\delta^{C}=\delta^{A}\cup\delta^{B}\cup\left\{ \left(q_{0}^{C},\textrm{�},q_{0}^{B}\right),\left(q_{f}^{B},\textrm{�},q_{f}^{C}\right)\right\} $.
\end{itemize}
In this construction, we allow to go from the initial state of $A$
to the one of $B$, and to go from the final state of $B$ to the
one of $A$. Thus, given a string $\mathbf{s}$, $C$ can span exactly
the $\left(V,\mathbf{s}\right)-$tuples contained in $A\left(\mathbf{s}\right)\cup B\left(\mathbf{s}\right)$.
Regarding the complexity, the operations that we perform here are
the union of the states sets and transition functions, with the addition
of two new transitions. Let us consider the size of the input as $n=n_{A}+n_{B}=\left|Q^{A}\right|+\left|\delta^{A}\right|+\left|\textrm{SVars}\left(A\right)\right|+\left|Q^{B}\right|+\left|\delta^{B}\right|+\left|\textrm{SVars}\left(B\right)\right|$.
If we assume the costs of basic operations as adding/deleting a state/transitions
to be constant, then obtaining $Q^{C}$ takes $O\left(\left|Q^{B}\right|\right)=O\left(n\right)$
time, and constructing $\delta^{C}$ takes $O\left(\left|\delta^{B}\right|\right)=O\left(n\right)$
time.
\end{proof}


\begin{theorem}
\noindent Given a well-behaved extended vset-automaton $A$, a set
$X=\textrm{SVars}\left(A\right)$ and a set $Y\subseteq X$, a vset-automaton
$A'$ can be produced in polynomial time such that $\left\llbracket A'\right\rrbracket =\pi_{Y}\left\llbracket A\right\rrbracket $,
in a well-behavedness preserving manner.\end{theorem}

\begin{proof}
Let us consider $A=\left(Q,q_{0},q_{f},\delta\right)$. We can take
$A'=\left(Q',q_{0}',q_{f}',\delta'\right)$, where:
\begin{itemize}
\item $Q'=Q$;
\item $q_{0}'=q_{0}$;
\item $q_{f}'=q_{f}$;
\item $\delta'=\delta\setminus\delta^{\textrm{unprojected}}\cup\delta^{\textrm{projected}}$,
where:

\begin{itemize}
\item $\delta^{\textrm{unprojected}}=\left\{ \left(q,S,q'\right)\in\delta\mid S\cap\textrm{SVOps}\left(X\setminus Y\right)\neq\textrm{�}\right\} $;
\item $\delta^{\textrm{projected}}=\left\{ \left(q,S',q'\right)\mid\exists\left(q,S,q'\right)\in\delta^{\textrm{unprojected}}:S'=S\setminus\textrm{SVOps}\left(X\setminus Y\right)\right\} $.
\end{itemize}
\end{itemize}
This construction removes all the occurencies of variable operations
that concern variables excluded from the projection in the transitions
of $A$. Since $A$ is well behaved, we have the guarantee that $A'=A$.
If this wasn't the case, our construction might result in $A\subseteq A'$.
$A'$ is still well-behaved, since we maintain the occurrencies of
operations concerning the variables on which we project, that continue
to appear exactly once on each path. With the usual assumptions, this
construction can be carried out in $O\left(n\right)$.
\end{proof}


\begin{definition}
An AQL core fragment is an AQL fragment that can be associated with
a core spanner.
\end{definition}


\begin{theorem}
Given a regex formula $\gamma$, $\gamma$ can be converted into an
equivalent, well-behaved extended vset-automaton $A$ in polynomial
time.\end{theorem}


\end{document}
