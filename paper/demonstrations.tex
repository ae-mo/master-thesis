%% LyX 2.1.4 created this file.  For more info, see http://www.lyx.org/.
%% Do not edit unless you really know what you are doing.
\documentclass[11pt,american,draftfoot]{ulb_thesis}
\usepackage[T1]{fontenc}
\usepackage[latin9]{inputenc}
\usepackage{amsmath}
\usepackage{amssymb}

\makeatletter
%%%%%%%%%%%%%%%%%%%%%%%%%%%%%% Textclass specific LaTeX commands.
 \input{ulb_ntheorem.std}

\makeatother

\usepackage{babel}
\begin{document}
$\textrm{SVOps}^{\vdash}=\left\{ x\vdash\mid x\in\textrm{SVars}\right\} $

$\textrm{SVOps}^{\dashv}=\left\{ x\dashv\mid x\in\textrm{SVars}\right\} $
\begin{definition}
An extended \emph{variable-set automaton} (or \emph{eVset-automaton})
is a tuple $\left(Q,q_{0},q_{f},\delta\right)$, where:
\begin{itemize}
\item $Q$ is a finite set of states;
\item $q_{0}\in Q$ is the initial state;
\item $q_{f}\in Q$ is an accepting state;
\item $\delta$ is a finite transition relation consisting of triples, each
having one of these forms:

\begin{itemize}
\item $\left(q,\sigma,q'\right)$, where $\sigma\in\varSigma$;
\item $\left(q,S,q'\right)$, where $S\subseteq\left(\textrm{SVOps}^{\vdash}\cup\textrm{SVOps}^{\dashv}\right)$.
\end{itemize}
\end{itemize}
\end{definition}

\noindent 
\begin{definition}
\noindent For a string $\mathbf{s}=s_{1},\ldots,s_{n}$, a run $\rho=c_{0}\ldots c_{m}$
on $\mathbf{s}$ of a (extended) variable-set automaton $A$ is \emph{well-behaved}
if it satisfies the following conditions:
\begin{itemize}
\item $c_{j}=\left(q,V,Y,i\right)$ where $q\neq q_{f}$ for $j=0,\ldots,m-1$;
\item either:

\begin{itemize}
\item $c_{m}=\left(q,V,Y,i\right)$ where $q\neq q_{f}$;
\item $c_{m}=\left(q_{f},\textrm{�},\textrm{�},n+1\right)$.
\end{itemize}
\end{itemize}
\end{definition}


\begin{definition}
\noindent A (extended) variable-set automaton is well-behaved if,
for every string $\mathbf{s}\in\varSigma^{\ast}$, $\textrm{ARuns}\left(A,\mathbf{s}\right)$
contains only well-behaved runs.
\end{definition}

\noindent 
\begin{lemma}
\noindent Given an extended varable-set automaton $A$, $A$ can be
converted into an equivalent variable-set automaton $A'$ in polynomial
time.\end{lemma}

\begin{proof}
Consider $A'=$\end{proof}


\end{document}
