%% LyX 2.1.4 created this file.  For more info, see http://www.lyx.org/.
%% Do not edit unless you really know what you are doing.
\documentclass[11pt,american,draftfoot]{ulb_thesis}
\usepackage[T1]{fontenc}
\usepackage[latin9]{inputenc}
\usepackage{amsmath}
\usepackage{amssymb}

\makeatletter
%%%%%%%%%%%%%%%%%%%%%%%%%%%%%% Textclass specific LaTeX commands.
 \input{ulb_ntheorem.std}

\makeatother

\usepackage{babel}
\begin{document}
$\textrm{SVOps}^{\vdash}=\left\{ x\vdash\mid x\in\textrm{SVars}\right\} $

$\textrm{SVOps}^{\dashv}=\left\{ x\dashv\mid x\in\textrm{SVars}\right\} $

SVOps$=\textrm{SVOps}^{\vdash}\cup\textrm{SVOps}^{\dashv}$
\begin{definition}
An extended \emph{variable-set automaton} (or \emph{eVset-automaton})
is a tuple $\left(Q,q_{0},q_{f},\delta\right)$, where:
\begin{itemize}
\item $Q$ is a finite set of states;
\item $q_{0}\in Q$ is the initial state;
\item $q_{f}\in Q$ is an accepting state;
\item $\delta$ is a finite transition relation consisting of triples, each
having one of these forms:

\begin{itemize}
\item $\left(q,\sigma,q'\right)$, where $\sigma\in\varSigma$;
\item $\left(q,S,q'\right)$, where $S\subseteq\left(\textrm{SVOps}\right)$.
\end{itemize}
\end{itemize}
\end{definition}

\noindent 
\begin{definition}
\noindent For a string $\mathbf{s}=s_{1},\ldots,s_{n}$, a run $\rho=c_{0}\ldots c_{m}$
on $\mathbf{s}$ of a (extended) variable-set automaton $A$ is \emph{well-behaved}
if it satisfies the following conditions:
\begin{itemize}
\item $c_{j}=\left(q,V,Y,i\right)$ where $q\neq q_{f}$ for $j=0,\ldots,m-1$;
\item either:

\begin{itemize}
\item $c_{m}=\left(q,V,Y,i\right)$ where $q\neq q_{f}$;
\item $c_{m}=\left(q_{f},\textrm{�},\textrm{�},n+1\right)$.
\end{itemize}
\item for every $c_{j},c_{j+1}$ and $c_{k},c_{k+1}$, $j\neq k$, such
that $c_{j}$ yelds $c_{j+1}$ by applying transition $\left(q_{j},o,q_{j+1}\right)\in\delta$
and the same holds for $c_{k},c_{k+1}$ with transition $\left(q_{k},o',q_{k+1}\right)\in\delta$
($o,o'\in\textrm{SVops}$$\left(A\right)$), $o\neq o'$;
\end{itemize}
\end{definition}


\begin{definition}
\noindent A (extended) variable-set automaton is well-behaved if,
for every string $\mathbf{s}\in\varSigma^{\ast}$, $\textrm{ARuns}\left(A,\mathbf{s}\right)$
contains only well-behaved runs.
\end{definition}

\noindent 
\begin{lemma}
\noindent Given an extended well-behaved varable-set automaton $A$,
$A$ can be converted into an equivalent well-behaved variable-set
automaton $A'$ in polynomial time.\end{lemma}

\begin{proof}
Without loss of generality, we consider an ordering of the symbols
in $\textrm{SVOps}\left(A\right)$ of the following form:
\[
x\vdash,\ldots,y\vdash,x\dashv,\ldots,y\dashv
\]
 In this ordering, all insertion operations come before the deletion
operations. We define $o\prec o'$, $o,o'\in\textrm{SVOps}\left(A\right)$,
if $o$ comes before $o'$ (\emph{not} if they are equal) in the chosen
ordering. Consider $A'=$$\left(Q',q_{0}',q_{f}',\delta'\right)$,
with $\textrm{SVars}\left(A'\right)=\textrm{SVars}\left(A\right)$,
whose components are defined as follows:
\begin{itemize}
\item $Q'=Q\cup Q^{\textrm{ops}}\cup Q^{\textrm{�}}$, where:

\begin{itemize}
\item $Q^{\textrm{ops}}=\left\{ q_{q',o,q''}\mid\exists\left(q',S,q''\right)\in\delta:o\in S\right\} $;
\item $Q^{\textrm{�}}=\left\{ q_{\textrm{q',�,q''}}\mid\exists\left(q',\textrm{�},q''\right)\in\delta\right\} $;
\end{itemize}
\item $q_{0}'=q_{0};$
\item $q_{f}'=q_{f}$;
\item $\delta'=\left(\delta\setminus\delta^{S}\right)\cup\delta^{\textrm{ops}}\cup\delta^{\textrm{�}}\cup\delta^{\varepsilon}$,
where:

\begin{itemize}
\item $\delta^{S}=\left\{ \left(q,S,q'\right)\in\delta\right\} ;$
\item $\delta^{\textrm{ops}}=\left\{ \left(q_{q',o,q''},o,q_{q',o',q''}\right)\mid\exists\left(q',S,q''\right)\in\delta:\left(o,o'\epsilon S\land o\prec o'\land\forall o''\in S:o\nprec o''\nprec o'\right)\right\} \cup\left\{ \left(q_{q',o,q''},o,q''\right)\mid\exists\left(q',S,q''\right)\in\delta:\left(o\epsilon S\land\forall o'\in S:o\nprec o'\right)\right\} $;
\item $\delta^{\textrm{�}}=\left\{ \left(q_{\textrm{q',�,q''}},\varepsilon,q''\right)\mid\exists\left(q',\textrm{�},q''\right)\in\delta\right\} ;$
\item $\delta^{\varepsilon}=\left\{ \left(q,\varepsilon,q'\right)\mid\left(\exists\left(q',o,q''\right)\in\delta^{\textrm{ops}}:\forall\left(q',o',q''\right)\in\delta^{\textrm{ops}}:o'\nprec o\right)\lor\left(\exists\left(q',\varepsilon,q''\right)\in\delta^{\textrm{�}}\right)\right\} $.
\end{itemize}
\end{itemize}
This construction expands the transitions of $A$ that are labeled
with a set of variable operations into a sequence of transitions performing
one operation at a time, taking care of putting the insertion operations
before the deletion ones, thus preserving well-behavedness. It also
substitutes transitions labeled with the empty set with ordinary $\varepsilon-$transitions.
To prove equivalence between $A$ and $A'$ it is sufficient to notice
that for every string $\mathbf{s}$, every run $\rho=c_{0},\ldots,c_{m}$$\in\textrm{ARuns}\left(A,\mathbf{s}\right)$
can be put in correspondence with a run $\rho'\in\textrm{ARuns}\left(A',\mathbf{s}\right)$.
Indeed, for every pair of consecutive configurations $c_{j}=\left(q_{j},V_{j},Y_{j},i_{j}\right)$,
$c_{j+1}=\left(q_{j+1},V_{j+1},Y_{j+1},i_{j+1}\right)$:
\begin{itemize}
\item if $i_{j}=i_{j+1}$ and $\left(q,S,q'\right)$$\in\delta$:

\begin{itemize}
\item if $V_{j}\neq V_{j+1}$, replace $c_{j}$, $c_{j+1}$ with a sequence
$c_{j,1},\ldots,c_{j,k}$ of configurations, where:

\begin{itemize}
\item $c_{j,1}=c_{j}$;
\item $c_{j,k}=c_{j+1}$; 
\item $c_{j,2}=\left(q_{j,2},V_{j,1},Y_{j,1},i_{j}\right)$ ($\left(q_{j,1},\varepsilon,q_{j,2}\right)\in\delta^{\varepsilon})$;
\item for every pair $c_{j,l}=\left(q_{j,l},V_{j,l},Y_{j,l},i_{j}\right),c_{j,l+1}=\left(q_{j,l+1},V_{j,l+1},Y_{j,l+1},i_{j}\right)$,
different from $c_{j,1},c_{j,2}$, there exists a transition $\left(q_{j,l},o,q_{j,l+1}\right)\in\delta^{\textrm{ops}}$
such that $o\in S$ and $V_{j,l+1},Y_{j,l+1}$ are obtained from $V_{j,l},Y_{j,l}$
according to $o$;
\end{itemize}
\item if $V_{j}=V_{j+1}$, replace $c_{j}$, $c_{j+1}$with $c_{j,1},c_{j,2},c_{j,3}$,
where:

\begin{itemize}
\item $c_{j,1}$ and $c_{j,2}$ are as in the previous case;
\item $c_{j,3}=c_{j+1}$;
\item there exists a transition $\left(q_{j,2},\textrm{�},q_{j,3}\right)\in\delta^{\textrm{�}}$
and $V_{j,3}=V_{j,l}$, $Y_{j,3}=Y_{j,2}$;
\end{itemize}
\end{itemize}
\item if $i_{j+1}=i_{j}+1$, do nothing.
\end{itemize}
After having applied this modifications to $\rho$, the obtained run
$\rho'$ will be a valid run of $A'$ on $\mathbf{s}$ and it will
be accepting if and only if $\rho$ was accepting, in which case it
will return the same $\mathbf{s}-$tuple returned by $\rho$. This
is because the only occasions where we actually modify $\rho$ is
when we are performing variable operations between a configuration
and the next one. In this case, we are not advancing the string pointer.
The new series of configurations we put in place of the original pair
doesn't advance the pointer either, and it performs the exact variable
operations that were performed before the substitution, insertions
first, thanks to the way we constructed $A'$. Thus, for every pair
of configurations $c_{j},c_{j+1}\in\rho$ that we replace, $c_{j+2}$
won't see any difference. Also notice how the replaced sequence $c_{j,1},\ldots,c_{j,k}$
is unique, and the sequence of operations from $c_{j,1}$ to $c_{j,k}$
respect the operation ordering that we chose.\end{proof}


\end{document}
